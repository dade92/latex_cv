\documentclass[letterpaper]{twentysecondcv} % a4paper for A4

%----------------------------------------------------------------------------------------
%	 PERSONAL INFORMATION
%----------------------------------------------------------------------------------------

% IMPORTANT:If you don't need one or more of the below, just remove the content leaving the command, e.g. \cvnumberphone{}

\profilepic{fotocv.jpg} % Profile picture

\cvname{Davide Botti} % Your name
\cvjobtitle{Software Engineer} % Job title/career

\cvdate{September 11, 1992} % Date of birth
\cvaddress{Italy, Via Verdi 4D,\newline 
	Orsenigo (CO)} % Short address/location, use \newline if more than 1 line is required
\cvnumberphone{+39 3406983956} % Phone number
\cvsite{} % Personal website
\cvmail{davidebotti92@gmail.com} % Email address

\begin{document}

%----------------------------------------------------------------------------------------
%	 ABOUT ME
%----------------------------------------------------------------------------------------

\aboutme{I was born in Erba, a small town in the north of Italy, and I got fond of computers since high school, when I started to study C and then Java by myself. After graduation, it was clear for me that I had to improve my knowledge in the field of computer science, so I enrolled in the course of Computer Engineering at Politecnico di Milano, obtaining a master degree in 2017}
% To have no About Me section, just remove all the text and leave \aboutme{}

%----------------------------------------------------------------------------------------
%	 SKILLS
%----------------------------------------------------------------------------------------

% Skill bar section, each skill must have a value between 0 an 6 (float)
\skills{{ARM Assembly, LateX/1},{C, C++/2},{Python, MySQL, NoSQL, Docker/3}, {Javascript, Typescript, React, Html, CSS/4.5},{Java, Spring Framework, Maven, Kotlin/5}}

%------------------------------------------------

% Skill text section, each skill must have a value between 0 an 6
%\skillstext{{lovely/4},{narcissistic/3}}

%----------------------------------------------------------------------------------------

\makeprofile % Print the sidebar

%----------------------------------------------------------------------------------------
%	 EDUCATION
%----------------------------------------------------------------------------------------

\section{Education}

\begin{twenty} % Environment for a list with descriptions
	\twentyitem{2014-2017}{Master's degree in Computer Science and Engineering}{Milano}{110/110 \textit{cum laude}}
	\twentyitem{2011-2014}{Bachelor's degree in Computer Science and Engineering}{Como}{110/110}
	\twentyitem{2006-2011}{Scientific high school diploma}{Erba}{95/100}
	%\twentyitem{<dates>}{<title>}{<location>}{<description>}
\end{twenty}

\section{Foreign Languages}

\begin{twenty}
	\twentyitem{2011}{First, B2-level English Certificate}{Como}{B}
\end{twenty}

%----------------------------------------------------------------------------------------
%	 EXPERIENCE
%----------------------------------------------------------------------------------------

\section{Working Experience}

\begin{twenty}
	%\twentyitem{<dates>}{<title>}{<location>}{<description>}
	\twentyitem{2017 - now}{Software Engineer at Lastminute.com}{}{}
	\twentyitem{2017}{Temporary Math and Physics professor in high school}{}{}
\end{twenty}

\section{Tech competences (in a nutshell)}

During my professional experience, I' ve grown in a complete agile environment, using scrum methods like kanban/sprint and practices like Test Driven Development and pair programming. \newline 
About hard skills: I was born as a backend Java/Kotlin developer, shifting during time to a more T-shaped figure, studying technologies outside my first area of interest like frontend frameworks (\emph{React}) devOps tools (\emph{Docker}) and some basic concepts about orchestration systems (\emph{Kubernetes}), all technologies that I used or got in touch during my career at Lastminute.com to deploy in production a valuable product.\newline


\section{Github Projects}
\newline\newline
Here a list of my Github projects to better understand what are my technical area of expertise.

\begin{twenty}
	%\twentyitem{<dates>}{<title>}{<location>}{<description>}
	\twentyitem{2023}{Example of a Spring Boot application with other technologies}{Java, Kotlin, Docker, NoSQL}{\url{https://github.com/dade92/spring-example2}}
		
	\twentyitem{2022}{Example of a React application}{Typescript, React}{\url{https://github.com/dade92/react-example2}}
	
	\twentyitem{2021}{Example of a Spring Boot application with CI integration}{Java, Kotlin, Docker, MySQL, Jenkins}{\url{https://github.com/dade92/spring-example1}}
	
	\twentyitem{2017}{Master Thesis: Architectural characterization of \newline side channel leakage on ARM Cortex-A7}{C++,  Assembly}{\url{https://github.com/dade92/side_channel_attack_suite}}
	
	\twentyitem{2016}{MIDI file reproducer on board STM32f4 Discovery}{C}{\url{https://github.com/dade92/es_project}}
	
	\twentyitem{2016}{Big Data analysis of Google statistics}{Python, Spark}{\url{https://github.com/dade92/google-trace-data-analysis}}
	
\end{twenty}

\section{Other information}

During my spare time I play guitar and study music. I'm a passionate reader and I really like playing chess. About sports, I like trekking, mountain bike and kayaking.\newline

 %Besides, I'm involved in a local newscast called TgOZ, broadcast weekly on our page \url{https://www.facebook.com/tgoznews/}.
 
 %----------------------------------------------------------------------------------------
 %	 PUBLICATIONS
 %----------------------------------------------------------------------------------------
 
 %\section{Publications}
 
 %\begin{twentyshort} % Environment for a short list with no descriptions
 %	\twentyitemshort{1865}{Chapter One, Down the Rabbit Hole.}
 %	\twentyitemshort{1865}{Chapter Two, The Pool of Tears.}
 %	\twentyitemshort{1865}{Chapter Three,  The Caucus Race and a Long Tale.}
 %	\twentyitemshort{1865}{Chapter Four,  The Rabbit Sends a Little Bill.}
 %	\twentyitemshort{1865}{Chapter Five,  Advice from a Caterpillar.}
 %\twentyitemshort{<dates>}{<title/description>}
 %\end{twentyshort}
 
 %----------------------------------------------------------------------------------------
 %	 AWARDS
 %----------------------------------------------------------------------------------------
 
 %\section{Awards}
 
 %\begin{twentyshort} % Environment for a short list with no descriptions
 %	\twentyitemshort{1987}{All-Time Best Fantasy Novel.}
 %	\twentyitemshort{1998}{All-Time Best Fantasy Novel before 1990.}
 %\twentyitemshort{<dates>}{<title/description>}
 %\end{twentyshort}
 
%\subsection{Review}

%Alice approaches Wonderland as an anthropologist, but maintains a strong sense of noblesse oblige that comes with her class status. She has confidence in her social position, education, and the Victorian virtue of good manners. Alice has a feeling of entitlement, particularly when comparing herself to Mabel, whom she declares has a ``poky little house," and no toys. Additionally, she flaunts her limited information base with anyone who will listen and becomes increasingly obsessed with the importance of good manners as she deals with the rude creatures of Wonderland. Alice maintains a superior attitude and behaves with solicitous indulgence toward those she believes are less privileged.

%----------------------------------------------------------------------------------------
%	 SECOND PAGE EXAMPLE
%----------------------------------------------------------------------------------------

%\newpage % Start a new page

%\makeprofile % Print the sidebar

%\section{Other information}

%\subsection{Review}

%Alice approaches Wonderland as an anthropologist, but maintains a strong sense of noblesse oblige that comes with her class status. She has confidence in her social position, education, and the Victorian virtue of good manners. Alice has a feeling of entitlement, particularly when comparing herself to Mabel, whom she declares has a ``poky little house," and no toys. Additionally, she flaunts her limited information base with anyone who will listen and becomes increasingly obsessed with the importance of good manners as she deals with the rude creatures of Wonderland. Alice maintains a superior attitude and behaves with solicitous indulgence toward those she believes are less privileged.

%\section{Other information}

%\subsection{Review}

%Alice approaches Wonderland as an anthropologist, but maintains a strong sense of noblesse oblige that comes with her class status. She has confidence in her social position, education, and the Victorian virtue of good manners. Alice has a feeling of entitlement, particularly when comparing herself to Mabel, whom she declares has a ``poky little house," and no toys. Additionally, she flaunts her limited information base with anyone who will listen and becomes increasingly obsessed with the importance of good manners as she deals with the rude creatures of Wonderland. Alice maintains a superior attitude and behaves with solicitous indulgence toward those she believes are less privileged.

%----------------------------------------------------------------------------------------

\end{document} 
